% در این فایل، دستورها و تنظیمات مورد نیاز، آورده شده است.
%-------------------------------------------------------------------------------------------------------------------%
%%%شعر فارسی
\usepackage{multirow}
\usepackage{bidipoem}
\usepackage{algorithm,algorithmic}
\usepackage{multicol}
%%%%%%%%%%%%%%%%%%%%%%%5
%تبدیل نقطه به ممیز 
%\TXTmath
\usepackage[titles]{tocloft} 
\usepackage{times}
%%%%%%%%%%%%%%%%%%%%%%%%%%%%%%

%نوشتن اعداد یونانی
\makeatletter
\newcommand*{\rom}[1]{\expandafter\@slowromancap\romannumeral #1@}
\makeatother
%%%%%%%%%%%%%%%%%%%%%%%%%%%%%%%%%%%%%%%%%%%%%%%%%%%%%%%
% برای اینکه در عنوان بخش پاورقی بنویسیم
\usepackage{bidiftnxtra}
%%%%%%%%%%%%%%%%%%%%%%%%%%%%%%%%
%برای نوشتن بخش و... عدد را می‌توان عوض کرد
\setcounter{tocdepth}{3}
%%%%%%%%%%%%%%%%%%%%%%%%%%%%%%%%%%%%
\usepackage{sectsty}
%برای این که شماره زیر نویس ها در هر صفحه از یک شروع شود.
%\usepackage{perpage}
%\MakePerPage{footnote}
%%%%%%%%%%%%%%%%%%%%%%%%%%%%%%%%%%%%%%%%%%%%
%برای خط کشیدن خط افقی و ساده کردن اعداد
\newcommand \hcancel[2][black]{\setbox 0=\hbox{#2}%
 \rlap {\raisebox {.45 \ht 0}{\textcolor{#1}{\rule{\wd 0}{1pt}}}}#2}
 %%%%%%%%%%%%%%%%%%%%%%%%%%%%%%%%%%%%%%%%%%
 %برای رنگی نوشتن عبارت
\usepackage{color}
%%%%%%%%%%%%%%%%%%%%%%%%%%%%%%%%%%%%%%%
%از این برای خط کشیدن روی عبارت استفاده می‌شود.
\usepackage[thicklines]{cancel}
%%%%%%%%%%%%%%%%%%%%%%%%%%%%%%%%%%%%%%%%%%
%از این برای این که خط رنگی باشد استفاده می‌شود.اگر باشد همه خط ها قرمز وگرنه مشکی می‌شوند.
%\renewcommand{\CancelColor}{\color{red}}
%%%%%%%%%%%%%%%%%%%%%%%%%%%%%%%%%%%%%%%%%%%%
%برای نوشتن ماتریس که کنار و بالاش نوشته باشه
\usepackage{blkarray} 
%%%%%%%%%%%%%%%%%%%%%%%%%%%%%%%%%%%%%%%
% در ورژن جدید زی‌پرشین برای تایپ متن‌های ریاضی، این سه بسته، حتماً باید فراخوانی شود
\usepackage{amsthm,amssymb,amsmath}
%%%%%%%%%%%%%%%%%%%%%%%%%%%%%%%%%%%%%
% بسته‌ای برای تنطیم حاشیه‌های بالا، پایین، چپ و راست صفحه
\usepackage[top=40mm, bottom=40mm, left=25mm, right=35mm]{geometry}
%%%%%%%%%%%%%%%%%%%%%%%%%%%%%%%%%%%%%%%
% بسته‌‌ای برای ظاهر شدن شکل‌ها و تصاویر متن
\usepackage{graphicx}
%%%%%%%%%%%%%%%%%%%%%%%%%%%%%%%%%%%%%%%%%%
% بسته‌ای برای رسم کادر
\usepackage{framed} 
%%%%%%%%%%%%%%%%%%%%%%%%%%%%%%%%%%%%%
% بسته‌‌ای برای چاپ شدن خودکار تعداد صفحات در صفحه «معرفی پایان‌نامه»
\usepackage{lastpage}
%\usepackage{Persian-bib}
%%%%%%%%%%%%%%%%%%%%%%%%%%%%%%%%%%%%%%
% بسته‌‌ای برای ایجاد دیاگرام‌های مختلف
\usepackage[all]{xy}
%%%%%%%%%%%%%%%%%%%%%%%%%%%%%%%%%%%%
% بسته‌ و دستوراتی برای ایجاد لینک‌های رنگی با امکان جهش
\usepackage[pagebackref=false,colorlinks,linkcolor=blue,citecolor=magenta]{hyperref}
%%%%%%%%%%%%%%%%%%%%%%%%%%%%%%%%%%%%%%%%
% چنانچه قصد پرینت گرفتن نوشته خود را دارید، خط بالا را غیرفعال و  از دستور زیر استفاده کنید چون در صورت استفاده از دستور زیر‌‌، 
% لینک‌ها به رنگ سیاه ظاهر خواهند شد که برای پرینت گرفتن، مناسب‌تر است
\usepackage[pagebackref=false]{hyperref}
%%%%%%%%%%%%%%%%%%%%%%%%%%%%%%%%%%%%%%%%%%%
% بسته‌ لازم برای تنظیم سربرگ‌ها
\usepackage{fancyhdr}
%%%%%%%%%%%%%%%%%%%%%%%%%%%%%%%%%%%%%%%%%%%
% بسته‌ای برای ظاهر شدن «مراجع» و «نمایه» در فهرست مطالب
\usepackage[nottoc]{tocbibind}
%%%%%%%%%%%%%%%%%%%%%%%%%%%%%%%%%%%%%%%%%%%
% دستورات مربوط به ایجاد نمایه
\usepackage{makeidx}
\makeindex
%%%%%%%%%%%%%%%%%%%%%%%%%%%%%%%%%%%%%%%%%
%دستوری برای این که مربع آخر برهان توپر باشد. اگر غیر فعال کنی مربع تو خالی می‌شود.
%\renewcommand\qedsymbol{\blacksquare}
%%%%%%%%%%%%%%%%%%%%%%%%%%%%%%%%%%%%%%%%%%
%برای خط چین 
\usepackage{arydshln}
%%%%%%%%%%%%%%%%%%%%%%%%%%%%%%%%%%%%%%%%%
% فراخوانی بسته زی‌پرشین و تعریف قلم فارسی و انگلیسی
\usepackage{xepersian}
%%%%%%%%%%%%%%%%%%%%%%%%%%%%%%%%%%%%%%%%%%% 
%برای اینکه در ماتریس بتوان خطوط عمودی بین چند سطر کشید.
\settextfont[Scale=1.1]{XB Niloofar}
% از revision 118 زی‌پرشین به بعد، وارد کردن دستور زیر لازم نیست. توجه داشته باشید که در صورت  غیرفعال کردن این دستور،
% از فونت پیش‌فرض لاتک برای کلمات انگلیسی استفاده خواهد شد.
\setlatintextfont[ExternalLocation,BoldFont={lmroman10-bold},BoldItalicFont={lmroman10-bolditalic},ItalicFont={lmroman10-italic}]{lmroman10-regular}
%%%%%%%%%%%%%%%%%%%%%%%%%%
% چنانچه می‌خواهید اعداد در فرمول‌ها، انگلیسی باشد، خط زیر را غیرفعال کنید
%\setdigitfont[Scale=1.1]{Yas}
%برای این که در شماره صفحات، قضیه و ... صفر تو خالی بنویسد.
\settextfont{Yas}
\setdigitfont{Yas}
\DefaultMathsDigits
%\settextfont[Scale=1.1]{Yas}
%%%%%%%%%%%%%%%%%%%%%%%%%%
% تعریف قلم‌های فارسی و انگلیسی اضافی برای استفاده در بعضی از قسمت‌های متن
\defpersianfont\nastaliq[Scale=2]{IranNastaliq}
\defpersianfont\chapternumber[Scale=1.5]{XB Niloofar}
\defpersianfont\titr[Scale=1]{XB Titre}
%%%%%%%%%%%%%%%%%%%%%%%%%%
%%%%%%%%%%%%%%%%%%%%%%%%%%%%%%%%%%%%%%%%%%%%%%%%%%%%%%%%%%%%%%%%%%%%%%%%%%%%%%%%%%%%%
% دستوری برای حذف کلمه «چکیده»
\renewcommand{\abstractname}{}
% دستوری برای حذف کلمه «abstract»
\renewcommand{\latinabstract}{}
% دستوری برای تغییر نام کلمه «اثبات» به «برهان»
\renewcommand\proofname{\textbf{برهان}}
% دستوری برای تغییر نام کلمه «کتاب‌نامه» به «مراجع»
\renewcommand{\bibname}{مراجع}
% دستوری برای تعریف واژه‌نامه انگلیسی به فارسی
\newcommand\persiangloss[2]{#1\dotfill\lr{#2}\\}
% دستوری برای تعریف واژه‌نامه فارسی به انگلیسی 
\newcommand\englishgloss[2]{#2\dotfill\lr{#1}\\}
%%%%%%%%%%%%%%%%%%%%%%%%%%%%%%%%%%%%%%%%%%%
% تعریف و نحوه ظاهر شدن عنوان قضیه‌ها، تعریف‌ها، مثال‌ها و ...
\theoremstyle{definition}
\newtheorem{definition}{تعریف}[section]
\newtheorem{Definition}{Definition}[section]
\theoremstyle{theorem}
\newtheorem{theorem}[definition]{قضیه}
\newtheorem{lemma}[definition]{لم}
\newtheorem{proposition}[definition]{گزاره}
\newtheorem{corollary}[definition]{نتیجه}
\newtheorem{remark}[definition]{توجه}
\theoremstyle{definition}
\newtheorem{ex}[definition]{مثال}
\newtheorem{exa}[definition]{Example}
\newtheorem{y}[definition]{یادآوری}
\newtheorem{note}[definition]{نکته}
\newtheorem{tab}[definition]{تبصره}
%%%%%%%%%%%%%%%%%%%%%%%%%%%%%%%%%%%%%%%%%
% تعریف دستورات جدید برای خلاصه نویسی و راحتی کار در هنگام تایپ فرمول‌های ریاضی
\newcommand{\bR}{\mathbb{R}}
\newcommand{\cB}{\mathcal{B}}
\newcommand{\cO}{\mathcal{O}}
\newcommand{\cG}{\mathcal{G}}
\newcommand{\cU}{\mathcal{U}}
\newcommand{\cK}{\mathcal{K}}
\newcommand{\cS}{\mathcal{S}}
\newcommand{\rM}{\mathrm{M}}
\newcommand{\rC}{\mathrm{C}}
\newcommand{\rV}{\mathrm{V}}
\newcommand{\ls}{\mathrm{LSC}_{+}(X)}
\newcommand{\ce}{\mathrm{C}^{*}(X)}
\newcommand{\lsc}{\mathrm{LSC}}
\newcommand{\fB}{\mathfrak{B}}
\newcommand{\fM}{\mathfrak{M}}
\newcommand{\bt}{\begin{theorem}}
\newcommand{\et}{\end{theorem}}
\newcommand{\bl}{\begin{lemma}}
\newcommand{\el}{\end{lemma}}
\newcommand{\bc}{\begin{corollary}}
\newcommand{\ec}{\end{corollary}}
\newcommand{\bp}{\begin{proof}}
\newcommand{\ep}{\end{proof}}
\newcommand{\be}{\begin{example}}
\newcommand{\ee}{\end{example}}
\newcommand{\bd}{\begin{definition}}
\newcommand{\ed}{\end{definition}}
\newcommand{\ba}{\begin{align}}
\newcommand{\ea}{\end{align}}
\newcommand{\no}{\nonumber}
%%%%%%%%%%%%%%%%%%%%%%%%%%%%%%%%%%%%%%%%%%%
% دستورهایی برای سفارشی کردن سربرگ صفحات شماره صفحه بالای صفحه نوشته شود

\csname@twosidetrue\endcsname
\pagestyle{fancy}
\fancyhf{} 
\fancyhead[OL,EL]{\thepage}
\fancyhead[OR]{\small\rightmark}
\fancyhead[ER]{\small\leftmark}
\renewcommand{\chaptermark}[1]{%
\markboth{\thechapter.\ #1}{}}

%برای این که شماره صفحات پایین نوشته شود.
%\csname@twosidetrue\endcsname
%\pagestyle{fancy}
%\fancyhf{} 
%\fancyhead[OL,EL]{}
%\cfoot{\thepage}
%\fancyhead[OR]{\small\rightmark}
%\fancyhead[ER]{\small\leftmark}
%\renewcommand{\chaptermark}[1]{%
%\markboth{\thechapter.\ #1}{}}
%%%%%%%%%%%%%%%%%%%%%%%%%%%%%
% دستورهایی برای سفارشی کردن صفحات اول فصل‌ها
\makeatletter
\newcommand\mycustomraggedright{%
 \if@RTL\raggedleft%
 \else\raggedright%
 \fi}
\def\@part[#1]#2{%
\ifnum \c@secnumdepth >-2\relax
\refstepcounter{part}%
\addcontentsline{toc}{part}{\thepart\hspace{1em}#1}%
\else
\addcontentsline{toc}{part}{#1}%
\fi
\markboth{}{}%
{\centering
\interlinepenalty \@M
\ifnum \c@secnumdepth >-2\relax
 \huge\bfseries \partname\nobreakspace\thepart
\par
\vskip 20\p@
\fi
\huge\bfseries #2\par}%
\@endpart}
\def\@makechapterhead#1{%
\vspace*{200\p@}%
{\parindent \z@ \mycustomraggedright %\@mycustomfont
\ifnum \c@secnumdepth >\m@ne
\if@mainmatter

\Huge\bfseries \@chapapp\space {\chapternumber\thechapter}
\par\nobreak
\vskip 20\p@
\fi
\fi
\interlinepenalty\@M 
\huge \bfseries #1\par\nobreak
%\vskip 120\p@
\newpage}}
\makeatother